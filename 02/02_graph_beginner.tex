\documentclass[dvipdfmx]{beamer}
\usetheme{metropolis}
\setbeamertemplate{navigation symbols}{}
\usepackage{fancyvrb}
\usepackage{xcolor}
\usepackage{ulem}
\usepackage{pxjahyper}
\usepackage[deluxe, uplatex]{otf}
\usepackage{algorithm}
\usepackage[noend]{algorithmic}
\hypersetup{
  pdftitle={グラフアルゴリズム 初級},
  pdfauthor={Jun Yoneyama}
}
\usefonttheme{professionalfonts}
\AtBeginSection[]{}
\renewcommand{\kanjifamilydefault}{\gtdefault}

\title{グラフアルゴリズム 初級}
\author{Jun Yoneyama}
\date{\today}

\begin{document}

\frame{\maketitle}

\section{グラフとは}

\begin{frame}{\insertsection}
  \begin{block}{グラフとは何か}
    \begin{itemize}
      \item 有向グラフ
      \item 無向グラフ
    \end{itemize}
    の定義
  \end{block}
\end{frame}

\subsection{有向グラフ}

\begin{frame}{\insertsection: \insertsubsection}
  \begin{definition}
    以下を満たす $G = (V, E)$ を\emph{有向グラフ}という。
    \begin{itemize}
      \item $V$: 集合
      \item $E \subseteq V \times V$
    \end{itemize}
    このとき、$V$の元を\emph{頂点}、$E$の元を\emph{辺}という。
  \end{definition}
  \pause
  \alert{とりあえずこの定義を理解する必要はない}
\end{frame}

\end{document}
