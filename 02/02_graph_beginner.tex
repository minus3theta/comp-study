\documentclass[dvipdfmx]{beamer}
\usetheme{metropolis}
\setbeamertemplate{navigation symbols}{}
\usepackage{fancyvrb}
\usepackage{xcolor}
\usepackage{ulem}
\usepackage{pxjahyper}
\usepackage[deluxe, uplatex]{otf}
\usepackage{algorithm}
\usepackage[noend]{algorithmic}
\hypersetup{
  pdftitle={グラフアルゴリズム 初級},
  pdfauthor={Jun Yoneyama}
}
\usefonttheme{professionalfonts}
\AtBeginSection[]{}
\renewcommand{\kanjifamilydefault}{\gtdefault}

\title{グラフアルゴリズム 初級}
\author{Jun Yoneyama}
\date{\today}

\begin{document}

\frame{\maketitle}

\section{グラフとは}

\begin{frame}{\insertsection}
  \begin{block}{グラフとは何か}
    \begin{itemize}
      \item 有向グラフ
      \item 無向グラフ
    \end{itemize}
    の定義
  \end{block}
\end{frame}

\subsection{有向グラフ}

\begin{frame}{\insertsection: \insertsubsection}
  \begin{block}{有向グラフ}
    \centering
    \includegraphics[width=0.6\textwidth]{res/digraph-crp.pdf}
  \end{block}
\end{frame}

\begin{frame}{\insertsection: \insertsubsection}
  \begin{definition}
    以下を満たす $G = (V, E)$ を\emph{有向グラフ}という。
    \begin{itemize}
      \item $V$: 集合
      \item $E \subseteq V \times V$
    \end{itemize}
    このとき、$V$の元を\emph{頂点}、$E$の元を\emph{辺}という。
  \end{definition}
\end{frame}

\subsection{無向グラフ}

\begin{frame}{\insertsection: \insertsubsection}
  \begin{block}{無向グラフ}
    \centering
    \includegraphics[width=0.6\textwidth]{res/undirectedgraph-crp.pdf}
  \end{block}
\end{frame}

\begin{frame}{\insertsection: \insertsubsection}
  \begin{definition}
    以下を満たす $G = (V, E)$ を\emph{無向グラフ}という。
    \begin{itemize}
      \item $V$: 集合
      \item $E \subseteq \left\{ \{u, v\} \mid u, v \in V, u \neq v \right\}$
    \end{itemize}
    このとき、$V$の元を\emph{頂点}、$E$の元を\emph{辺}という。
  \end{definition}
\end{frame}

\begin{frame}{\insertsection: \insertsubsection}
  \begin{block}{無向グラフから有向グラフへの変換}
    以下を同一視して解釈すると上手くいく場合がある

    上手くいかない場合もある
    \begin{itemize}
      \item $u$と$v$を結ぶ無向辺がある
      \item $u$から$v$への有向辺と$v$から$u$への有向辺がある
    \end{itemize}
    \begin{figure}
      \centering
      \includegraphics{res/ud2d-ud-crp.pdf}
    \end{figure}
    \begin{figure}
      \centering
      \includegraphics{res/ud2d-d-crp.pdf}
    \end{figure}
  \end{block}
\end{frame}

\subsection{グラフの用語}

\begin{frame}{\insertsection: \insertsubsection}
  \begin{block}{\insertsubsection}
    \begin{itemize}
      \item ループ
      \item 多重辺
      \item 閉路
      \item 連結成分
      \item 連結
      \item 木
      \item $\vdots$
    \end{itemize}
    本講座では、出てきたときに解説する
  \end{block}
\end{frame}

\section{グラフの表現}

\begin{frame}{\insertsection}
  グラフをプログラム内で表現するときのデータ構造
  \begin{block}{例}
    \centering
    \includegraphics[width=0.4\textwidth]{res/triangle-crp.pdf}
  \end{block}
  \begin{columns}[t]
    \begin{column}{0.48\textwidth}
      \begin{block}{隣接行列}
        \begin{table}
          \begin{tabular}{c|ccc}
            i \ \textbackslash \ j & 0 & 1 & 2 \\ \hline
            0 & false & true & true \\
            1 & false & false & true \\
            2 & false & false & false
          \end{tabular}
        \end{table}
      \end{block}
    \end{column}
    \begin{column}{0.48\textwidth}
      \begin{block}{隣接リスト}
        \begin{table}
          \begin{tabular}{r|l}
            i & 隣接頂点 \\ \hline
            0 & [1, 2] \\
            1 & [2] \\
            2 & []
          \end{tabular}
        \end{table}
      \end{block}
    \end{column}
  \end{columns}
\end{frame}

\end{document}
